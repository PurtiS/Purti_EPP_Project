\documentclass[11pt, aspectratio=169]{beamer}
% \documentclass[11pt,handout]{beamer}
\usepackage[T1]{fontenc}
\usepackage[utf8]{inputenc}
\usepackage{textcomp}
\usepackage{float, afterpage, rotating, graphicx}
\usepackage{epstopdf}
\usepackage{longtable, booktabs, tabularx}
\usepackage{fancyvrb, moreverb, relsize}
\usepackage{eurosym, calc}
\usepackage{amsmath, amssymb, amsfonts, amsthm, bm}
\usepackage[
    natbib=true,
    bibencoding=inputenc,
    bibstyle=authoryear-ibid,
    citestyle=authoryear-comp,
    maxcitenames=3,
    maxbibnames=10,
    useprefix=false,
    sortcites=true,
    backend=biber
]{biblatex}
\AtBeginDocument{\toggletrue{blx@useprefix}}
\AtBeginBibliography{\togglefalse{blx@useprefix}}
\setlength{\bibitemsep}{1.5ex}
\addbibresource{refs.bib}

\hypersetup{colorlinks=true, linkcolor=black, anchorcolor=black, citecolor=black, filecolor=black, menucolor=black, runcolor=black, urlcolor=black}

\setbeamertemplate{footline}[frame number]
\setbeamertemplate{navigation symbols}{}
\setbeamertemplate{frametitle}{\centering\vspace{1ex}\insertframetitle\par}


\begin{document}

\title{Does unemployment lead to loneliness?}

\author[Purti Sadhwani]
{
{\bf Purti Sadhwani}\\
{\small University of Bonn}\\[1ex]
}


\begin{frame}[t]
    \frametitle{Introduction}
    \begin{itemize}
    \item This paper examines loneliness and its relationship with unemployment. More specifically,
    whether unemployment is a causal explanation for feelings of loneliness. The structure is as
    follows: this section gives an overview of the literature surrounding the consequences and
    causes of loneliness and motivates why the link with unemployment should be of interest.
    Section 2 describes the chosen dataset. Section 3.1 explains the issues of causal interpretation when identifying the impact of unemployment and 3.2 outlines the chosen empirical
    approach. Section 4 presents these results and finally section 5 provides discussion on scope
    to expand on the results and critiques of our approach.
    \end{itemize}
    \note{~}
\end{frame}

\begin{frame}[t]
    \frametitle{Data}
     \begin{itemize}
           \item SOEP contains several types of questions, which provide us with three different types
            of variables. Answers to questions about unemployment duration, household income, age
            and years of education are contained in numerical variables. Categorical variables indicate
            participants gender, whether they live in east Germany or not, and employment status. At
            last, we have the variables collected from a Likert scale answer, producing ordinal numbers. The questionnaire asks of the degree they would rate their health. It also asks three
            questions related to the subject of loneliness: how often they feel left alone, socially isolated
            and that the company of others is missing. All scored in a 1 (Bad / Very often) to 5 (Good
            / Never) interval. We generate an aggregate loneliness variable, taking the average of all
            three answers. For the rest of this paper, we emphasise that low levels of the loneliness
            variable, represents more frequent feelings of loneliness.
     \end{itemize}
    \note{~}
\end{frame}

\begin{frame}[t]
    \frametitle{Empirical Analysis}
     \begin{itemize}
          \item Our approach in order to isolate the impact of unemployment on loneliness is to use PSM.
          \item PSM attempts to emulate randomisation of treatment, by conditioning on selected observable characteristics. Thus, the untreated individuals can be compared to the treated with
            respect to the control variables.
            There are two important assumptions that need to be satisfied, in order for PSM to
            give a causal interpretation. The first being unconfoundedness,
            implying that the covariates should be independent of the treatment or the anticipation
            of the treatment. The second being the overlap condition, which makes sure that there is
            sufficient overlap between the covariates of the treatment and the control group. Together
            these conditions secure the strong ignorability treatment assignment, which states that
            after matching, the treatment variable is independent of the covariates.
            In order to use PSM, our empirical process must satisfy the mentioned assumptions.
            First, the sample is limited to only include those who are employed in 2013.
     \end{itemize}
    \note{~}
\end{frame}

\begin{frame}[t]
    \frametitle{Conclusion and Discussion}
    \begin{itemize}
        \item Our results show that there is evidence of unemployment affecting loneliness levels, which
        should be relevant for policy making. We point out two implications of our result. The first
        being that experiencing unemployment could have a negative impact on loneliness. This
        could be worsened by the introduction of zero-hour contracts, which could increase the frequency of unemployment in the economy. Secondly, as echoed by the literature, loneliness
        is an important component in individuals' health. The detrimental effects of unemployment
        are emphasised and potentially exacerbated by the increasing loneliness levels which can
        have negative consequences on one's health. This would suggest that policymakers should
        increase their focus on aiding individuals to avoid unemployment, and during unemployment provide sufficient arenas for social interaction. However, it should be noted that the
        external validity of the results is limited to only countries with a similar economic outlook.
    \end{itemize}
    \note{~}
\end{frame}



\begin{frame}[t]
    \begin{figure}[H]

        \centering
        \includegraphics[width=0.5\textwidth]{../bld/python/figures/descriptive stats_1.png}

        \caption{\emph{Python:} Model predictions of the smoking probability over the
            lifetime. Each colored line represents a case where marital status is fixed to
            one of the values present in the data set.}
        \label{fig:python-predictions}

    \end{figure}
\end{frame}






% Print black screen only in presentation mode for finishing up.
\mode<beamer> {
    \beamersetaveragebackground{black}
    \begin{frame}
        \frametitle{}
    \end{frame}

    \beamersetaveragebackground{white}
}

\begin{frame}[allowframebreaks]
    \frametitle{References}
    \renewcommand{\bibfont}{\normalfont\footnotesize}
    \printbibliography
\end{frame}

\end{document}
